% Options for packages loaded elsewhere
\PassOptionsToPackage{unicode}{hyperref}
\PassOptionsToPackage{hyphens}{url}
%
\documentclass[
]{article}
\usepackage{amsmath,amssymb}
\usepackage{iftex}
\ifPDFTeX
\usepackage[T1]{fontenc}
\usepackage[utf8]{inputenc}
\usepackage{textcomp} % provide euro and other symbols
\usepackage{lmodern}

\else % if luatex or xetex
\usepackage{unicode-math} % this also loads fontspec
\defaultfontfeatures{Scale=MatchLowercase}
\defaultfontfeatures[\rmfamily]{Ligatures=TeX,Scale=1}
\fi
\usepackage{lmodern}
\ifPDFTeX\else
% xetex/luatex font selection
\fi
% Use upquote if available, for straight quotes in verbatim environments
\IfFileExists{upquote.sty}{\usepackage{upquote}}{}
\IfFileExists{microtype.sty}{% use microtype if available
	\usepackage[]{microtype}
	\UseMicrotypeSet[protrusion]{basicmath} % disable protrusion for tt fonts
}{}
\makeatletter
\@ifundefined{KOMAClassName}{% if non-KOMA class
	\IfFileExists{parskip.sty}{%
		\usepackage{parskip}
	}{% else
		\setlength{\parindent}{0pt}
		\setlength{\parskip}{6pt plus 2pt minus 1pt}}
}{% if KOMA class
	\KOMAoptions{parskip=half}}
\makeatother
\usepackage{xcolor}
\usepackage[margin=1in]{geometry}
\usepackage{graphicx}
\makeatletter
\def\maxwidth{\ifdim\Gin@nat@width>\linewidth\linewidth\else\Gin@nat@width\fi}
\def\maxheight{\ifdim\Gin@nat@height>\textheight\textheight\else\Gin@nat@height\fi}
\makeatother
% Scale images if necessary, so that they will not overflow the page
% margins by default, and it is still possible to overwrite the defaults
% using explicit options in \includegraphics[width, height, ...]{}
\setkeys{Gin}{width=\maxwidth,height=\maxheight,keepaspectratio}
% Set default figure placement to htbp
\makeatletter
\def\fps@figure{htbp}
\makeatother
\setlength{\emergencystretch}{3em} % prevent overfull lines
\providecommand{\tightlist}{%
	\setlength{\itemsep}{0pt}\setlength{\parskip}{0pt}}
\setcounter{secnumdepth}{5}
\usepackage{caption}
\usepackage{subcaption}
\usepackage{multirow}
\usepackage{float}
\restylefloat{table}
\let\oldtable\table
\let\endoldtable\endtable
\renewenvironment{table}[1][H]{\oldtable[H]}{\endoldtable}
\usepackage{fancyhdr}
\pagestyle{fancy}
\fancyhf{}
\fancyhead[L]{\textbf{2025-05-24}}
\fancyhead[C]{\textbf{Term Project in Econometrics of Time Series}}
\fancyhead[R]{\textbf{Gereon Staratschek}}
\fancyfoot[C]{\thepage}
\usepackage{booktabs}
\usepackage{longtable}
\usepackage{array}
\usepackage{multirow}
\usepackage{wrapfig}
\usepackage{float}
\usepackage{colortbl}
\usepackage{pdflscape}
\usepackage{tabu}
\usepackage{threeparttable}
\usepackage{threeparttablex}
\usepackage[normalem]{ulem}
\usepackage{makecell}
\usepackage{xcolor}

\usepackage{adjustbox} 


\ifLuaTeX
\usepackage{selnolig}  % disable illegal ligatures
\fi
\usepackage{bookmark}
\IfFileExists{xurl.sty}{\usepackage{xurl}}{} % add URL line breaks if available
\urlstyle{same}
\hypersetup{
	pdftitle={Macroeconometrics - UK variables report},
	pdfauthor={Souleymane Faye \& Gereon Staratschek},
	hidelinks,
	pdfcreator={LaTeX via pandoc}}

\captionsetup{font=large}

% Added by Souley
\usepackage{booktabs}
\usepackage{siunitx}
\usepackage{multirow}
\usepackage{threeparttable}
\usepackage{pdflscape}
\usepackage{booktabs}    % For professional table formatting
\usepackage{dcolumn}     % For decimal-aligned columns
\usepackage{geometry}    % Adjust page margins if needed
\geometry{a4paper, margin=1in}
\usepackage{amsmath} 
\usepackage{bm} 
\newtheorem{assumption}{Assumption} % Define assumption environment

\title{Term Project - Introduction to Econometrics of Time Series \\[1.5ex] 
	{\Large An analysis of United Kingdom's GDP, Trade Balance, and Exchange Rate, 1955-2024}}
\author{Souleymane \textsc{Faye} \& Gereon \textsc{Staratschek}}
\date{2025-05-24}

\begin{document}
	\maketitle
	
	\section{Introduction}
	
	In this project, we aim to analyze the development of the Gross Domestic
	Product (GDP), exchange rates, and trade balance of the United Kingdom
	(UK). We collect our data from the OECD open data portal. We use data on
	a quarterly basis, starting latest in 1997. This period covers important
	financial events such as the Global Financial Crisis (GFC) in 2007, the
	Brexit referendum in 2016 and UK's final EU leave in 2020 as well as the
	COVID-19 pandemic from 2020-2022. Section 2 presents the univariate analysis,
	while section 3 focuses on the multivariate analysis. Section 4 briefly concludes.
	
	\section{Univariate Analysis}
	
	\subsection{Macro trends}
	
	Figure 1 illustrates the evolution of the UK’s macroeconomic indicators—GDP, Trade Balance, 
	and Exchange Rate—in both levels and first differences.
	
	\paragraph*{Trend Dynamics and Structural Breaks in Level Series.} 
	Data on UK's GDP, available on a quarterly basis
	since 1955, exhibits a clear, upwards trajectory with only a few
	shocks such as the 2007 financial crisis or the 2020 COVID-19 pandemic 
	interrupting the general trend. Hence, the GDP of the UK is clearly not
	stationary. However, the trend seems to be linear, suggesting that the
	first-differences time series of the GDP might be stationary. Similarly, data on 
	the trade balance is available on a quarterly basis starting in 1955.We see that 
	the trade balance fluctuates around 0 until the 1990s where a negative trend seems to set
	in continueing until the 2010s, when trade balance starts fluctuating
	around a low, negative value. However, since the observed spikes are
	getting much bigger over time, we also see an increase in fluctuation
	around the respective stationary mean.
	
	\begin{figure}[H]
		\caption{\large \textsc{Trends in GDP, Trade Balance, and Exchange Rate}}\label{fig:trends-plot}
		
		{\centering \includegraphics[width=0.8\linewidth]{../figures/uk_macro_levels_and_differences.png} 
			
		}
		
		
		{\footnotesize {\textit{Notes}: This figure presents levels (top row) and 
				first differences (bottom row) of key UK macroeconomic indicators: real GDP 
				(left), trade balance (center), and exchange rate against the US dollar (right),
				from 1955 to 2023. All series are shown alongside Hodrick-Prescott (HP) trends 
				with a smoothing parameter of $\lambda=1600$.  \\ \textit{Source}: Authors' 
				computation from the UK Statistical Office.
		} }
		
	\end{figure}
	Hence, the data seems to be
	stationary in the beginning and in the end with a negative trend being
	observed between the 1990s and the 2010s. Data on the exchange rate is available since
	1997 on a quarterly basis. It exhibits stationarity between 1997 and
	2007, as well as from 2007 onwards. In 2007, a shock seems to have
	shifted the mean of the stationary process downwards. The bottom row highlights first-differenced series, which strip away trends to 
	reveal stationary fluctuations and cyclical patterns, supporting the idea that 
	differencing mitigates non-stationarity. Across all panels, Hodrick-Prescott 
	filtered trends ($\lambda=1600$, standard for quarterly data) visually disentangle 
	long-term trajectories from cyclical noise.
	
	\subsection{Unit roots and stationarity tests}
	
	In this section, we aim to formally conduct stationarity tests for the
	series. As explained in the last section, we have reason to doubt that
	our series are entirely stationary. However, for our analysis, we are
	relying on stationarity properties of the series. Hence, after
	identifying the non-stationary series formally, we will conduct the
	first-difference transformation to obtain stationary series for our
	analysis.
	
	Table 1 presents the results of unit root and stationarity tests for our variables of interest, 
	analyzed in both levels and first differences. The Augmented 
	Dickey-Fuller (ADF), Phillips-Perron (PP), and Elliott, Rothenberg and Stock (ERS DF-GLS) tests evaluate the 
	null hypothesis of a unit root (non-stationarity), while the Kwiatkowski-Phillips-Schmidt-Shin (KPSS) test assesses 
	the null of stationarity. For GDP, the level series exhibits non-stationarity 
	across all tests (e.g., ADF p-value = 0.353, KPSS statistic = 4.732*), but its 
	first-differenced series shows strong evidence of stationarity (ADF statistic =
	-8.020*, KPSS = 0.086). Similarly, the trade balance in levels displays 
	persistent non-stationarity (KPSS = 3.395*), with stationarity achieved after 
	first differencing (ADF = -8.171*). The exchange rate shows mixed results in 
	levels (e.g., ADF p-value = 0.418), but clear stationarity in differences
	(PP = -82.885***)
	
	\begin{table}[!htbp]
		\centering
		\caption{\textsc{Unit Root and Stationnary Test Results}}
		\label{tab:unit_root}
		\begin{tabular}{@{} l cccc @{}}
			\\[-1.8ex] \hline 
			\hline  \\[-1.8ex] 
			& \multicolumn{4}{c}{Test Statistics} \\
			\cmidrule(lr){2-5}
			Series/Test & ADF & PP & ERS DF-GLS & KPSS \\
			\midrule
			
			\multicolumn{5}{@{}l}{Gross Domestic Product} \\
			\cmidrule(lr){1-5}
			Levels & $-2.531^{}$ $(0.353)$ & $-21.533^{***}$ $(0.048)$ & $3.074$ & $4.732^{***}$ \\
			Differences & $-8.020^{***}$ $(0.010)$ & $-321.613^{***}$ $(0.010)$ & $-7.941^{***}$ & $0.086$ \\
			\addlinespace
			
			\multicolumn{5}{@{}l}{Trade Balance} \\
			\cmidrule(lr){1-5}
			Levels & $-2.225^{}$ $(0.481)$ & $-157.415^{***}$ $(0.010)$ & $-0.672$ & $3.395^{***}$ \\
			Differences & $-8.171^{***}$ $(0.010)$ & $-300.088^{***}$ $(0.010)$ & $-12.912^{***}$ & $0.035$ \\
			\addlinespace
			
			\multicolumn{5}{@{}l}{Exchange Rate} \\
			\cmidrule(lr){1-5}
			Levels & $-2.382^{}$ $(0.418)$ & $-13.158$ $(0.354)$ & $-1.415$ & $1.754^{***}$ \\
			Differences & $-5.041^{***}$ $(0.010)$ & $-82.885^{***}$ $(0.010)$ & $-2.188$ & $0.084$ \\
			\hline \hline 
		\end{tabular}
		
		\vspace{0.2cm}
		\begin{minipage}{\textwidth}
			\scriptsize
			\textit{Notes}: Null hypotheses—ADF/PP/ERS: series has a unit root 
			(non-stationary); KPSS: series is stationary. To establish stationarity: reject
			ADF/PP/ERS null (significant ***/**) \textit{and} fail to reject KPSS null 
			(statistic $<$ critical value). P-values in parentheses. Critical values (1\% level): ADF/PP = $-3.43$, 
			ERS DF-GLS = $-2.57$, KPSS = $0.739$. $^{***}p<0.01$, $^{**}p<0.05$. 
			First differences calculated as $\Delta y_t = y_t - y_{t-1}$. 
			
			\textit{Source}: Author's calculations using data from the UK Statistical Office.
		\end{minipage}
	\end{table}
	
	We highlight that first-differencing effectively mitigates
	non-stationarity, as evidenced by statistically significant rejections of
	unit root hypotheses (***p<0.01) and failure to reject KPSS stationarity for 
	differenced series.These results justify the use of differenced series for subsequent analysis, 
	ensuring compliance with the stationarity assumptions underlying any further 
	econometric treatment. Critical values and p-values are reported to validate the robustness
	of conclusions.
	
	\subsection{Model Estimation}
	
	Next, we proceed to the model estimation.
	
	\paragraph*{Statistical Models.}
	
	Denote \( \nabla y_t = y_t - y_{t-1} \) the first-difference operator. We write
	down three statistical models to guide our analysis. The univariate trade balance MA (4) model writes
	\begin{equation}
		\nabla tb_t = \theta_1 \epsilon_{t-1} + \theta_2 \epsilon_{t-2} + \theta_3 \epsilon_{t-3} + \theta_4 \epsilon_{t-4} + \epsilon_t
	\end{equation}
	where \( tb_t \) denotes trade balance and $(\epsilon_t)_{1,..., T}$ is the innovation process.
	
	The exchange rate AR (1) model is 
	\begin{equation}
		\nabla e_t = \phi_1 \nabla y_{t-1} + \nu_t
	\end{equation}
	where \( e_t \) represents exchange rate and $(\nu_t)_{1,..., T}$ is the innovation process 
	The GDP univariate ARIMA (1,1,4) model with drift writes
	\begin{equation}
		\nabla y_t = c + \phi_1 \nabla y_{t-1} + \sum_{i=1}^4 \theta_i \gamma_{t-i} + \gamma_t
	\end{equation}
	where \( y_t \) is GDP and $(\gamma_t)_{1,..., T}$ is the innovation process.
	
	\paragraph*{Estimation.} Table 2 shows the results of our estimations. We find that the Trade Balance 
	has significant MA terms at the 1\% level (\( \theta_1 = -0.775^{***},\ \theta_4 = 0.278^{***} \)) 
	indicating strong short- and long-lagged shock persistence. As for the exchange rate,
	The autoregressive term (\( \phi_1 = 0.259^{***} \)) suggests moderate persistence 
	in differenced exchange rate movements. GDP's key drivers include the AR(1)
	term (\( \phi_1 = 0.489^{**} \)) and MA terms (\( \theta_1 = -0.770^{***},\ 
	\theta_4 = -0.241^{***} \)), with a large intercept (\( c = 1832.987^{***} \)) 
	reflecting persistent upward drift. The models
	demonstrate heterogeneous dynamics: Trade balance exhibits complex error correction, 
	exchange rate shows momentum effects, and GDP combines trend persistence with 
	multi-period shock absorption. 
	
	\begin{table}[htbp]
		\centering
		\caption{\textsc{ARIMA Model Estimates}}
		\label{tab:arima}
		\small
		\begin{tabular}{@{} >{}l *{6}{S} @{}}
			\\[-1.8ex] \hline \hline \\[-1.8ex] 
			& \multicolumn{2}{c}{Trade Balance} & \multicolumn{2}{c}{Exchange Rate} & \multicolumn{2}{c}{GDP} \\
			\cmidrule(l{5pt}r{5pt}){2-3} \cmidrule(l{5pt}r{5pt}){4-5} \cmidrule(l{5pt}r{5pt}){6-7}
			Component & {Coeff.} & {SE} & {Coeff.} & {SE} & {Coeff.} & {SE} \\
			\midrule
			
			AR(1)              & {--}          & {--}        & 0.259$^{***}$  & 0.092        & 0.489$^{**}$    & 0.202        \\
			\addlinespace[0.5em]
			MA(1)              & -0.775$^{***}$ & 0.057       & {--}           & {--}         & -0.770$^{***}$  & 0.199        \\
			\addlinespace[0.5em]
			MA(2)              & -0.100        & 0.074       & {--}           & {--}         & 0.093           & 0.093        \\
			\addlinespace[0.5em]
			MA(3)              & -0.030        & 0.080       & {--}           & {--}         & 0.135$^{*}$     & 0.075        \\
			\addlinespace[0.5em]
			MA(4)              & 0.278$^{***}$  & 0.064       & {--}           & {--}         & -0.241$^{***}$  & 0.061        \\
			\addlinespace[0.5em]
			Intercept          & {--}          & {--}        & {--}           & {--}         & 1832.987$^{***}$ & 232.754      \\
			
			\hline \hline \\[-1.8ex] 
		\end{tabular}
		
		\vspace{0.4cm}
		\begin{minipage}{\textwidth}
			\footnotesize
			\textit{Notes:} Coefficient estimates (Coeff.) with standard errors (SE). 
			{--} = Component not included in model. Significance levels: $^{***}p<0.01$, $^{**}p<0.05$, $^{*}p<0.1$. 
			AR = Autoregressive term, MA = Moving Average term. GDP intercept in original units. 
			\textit{Source:} Author's calculations using UK Statistical Office data.
		\end{minipage}
	\end{table}
	
	\subsection{Forecasting}
	
	Table 3 reports forecast accuracy 
	metrics for the first-differenced GDP, Trade Balance, and Exchange Rate series, 
	calculated on the training set. The GDP series exhibits large absolute errors,
	with a Root Mean Squared Error (RMSE) of 9,021.2 and a Mean Absolute Error (MAE)
	of 3,505.4, while its Mean Absolute Scaled Error (MASE) of 0.77 suggests 
	comparative performance relative to a benchmark. For the Trade Balance, the 
	RMSE of 4,513.0 and MAE of 2,255.2 are accompanied by undefined percentage 
	errors (MPE and MAPE), likely due to zero-crossings in the differenced series.
	The Exchange Rate demonstrates smaller-scale errors, with an RMSE of 0.035 and 
	MAE of 0.027, alongside percentage errors exceeding 100\%. Residual diagnostics
	show near-zero first-order autocorrelation (ACF1) across all series, ranging 
	from -0.003 for GDP to 0.006 for the Trade Balance. The MASE values, all below 
	1, indicate that the models improve upon a naive forecast. These metrics reflect
	the distinct scaling and volatility profiles of the macroeconomic series under study.
	
	\begin{table}[htbp]
		\centering
		\caption{\textsc{Forecast Accuracy Metrics for First-Differenced Series}}
		\label{tab:accuracy}
		\footnotesize
		\begin{tabular}{@{} l *{2}{S[table-format=-3.1]} S[table-format=4.1] S[table-format=-3.1] S[table-format=3.1] S[table-format=1.3] S[table-format=-1.3] @{}}
			\\[-1.8ex]\hline \hline \\[-1.8ex] 
			\textbf{Series} & \multicolumn{1}{c}{\textbf{ME}} & \multicolumn{1}{c}{\textbf{RMSE}} & \multicolumn{1}{c}{\textbf{MAE}} & \multicolumn{1}{c}{\textbf{MPE}} & \multicolumn{1}{c}{\textbf{MAPE}} & \multicolumn{1}{c}{\textbf{MASE}} & \multicolumn{1}{c}{\textbf{ACF1}} \\
			\midrule
			GDP & -30.6 & 9021.2 & 3505.4 & -106.5 & 452.6 & 0.770 & -0.003 \\[2.5ex] 
			Trade Balance & -196.1 & 4513.0 & 2255.2 & {--} & {--} & 0.686 & 0.006 \\[2.5ex] 
			Exchange Rate & -0.001 & 0.035 & 0.027 & 137.8 & 152.9 & 0.682 & -0.008 \\
			\hline \hline \\[-1.8ex] 
		\end{tabular}
		
		\vspace{0.2cm}
		\begin{minipage}{\textwidth}
			\scriptsize
			\textit{Notes:} All metrics calculated on training set. Scaling factors: 
			GDP RMSE/MAE in original units ($x10^{-3}$), Exchange Rate ME ($x10^{-3}$). 
			"{--}" indicates infinite values due to zero-base percentage calculations. 
			ME: Mean Error, RMSE: Root Mean Squared Error, MAE: Mean Absolute Error, 
			MPE: Mean Percentage Error, MAPE: Mean Absolute Percentage Error, 
			MASE: Mean Absolute Scaled Error, ACF1: Lag-1 Autocorrelation.
			
			\textit{Sources:} XX.
		\end{minipage}
	\end{table}
	
	\paragraph*{Forecast Performance Analysis.}  Figure \ref{fig:forecast_results} 
	evaluates the in-sample fit and out-of-sample forecasts of ARIMA models for
	UK macroeconomic variables. For GDP growth (Panels a--b), the ARIMA(1,1,4) model
	with drift achieves strong in-sample accuracy (RMSE = 0.027; MAE = 0.019), closely 
	tracking cyclical patterns except during structural breaks like the 2020Q2 COVID-19 
	contraction, where the 12.8\% quarterly decline exceeded the 
	model's $\pm$3.1\% prediction interval. Elevated percentage errors
	(MPE = 137.8\%, MAPE = 152.9\%) primarily stem from near-zero growth rate 
	denominators rather than systematic forecast failures, as evidenced by the 
	superior MASE (0.68) relative to a naive benchmark.  
	
	\begin{figure}[htbp]
		\centering
		\caption{\textsc{Forecast performance for UK GDP, Exchange Rate, and Trade Balance}}
		\label{fig:forecast_results}
		% GDP: In-sample and Forecast
		\begin{subfigure}[b]{0.45\textwidth}
			\includegraphics[width=\textwidth]{../figures/GDP_fitted_vs_actual.png}
			\caption{GDP: In-sample forecast}
			\label{fig:gdp_in}
		\end{subfigure}
		\hfill
		\begin{subfigure}[b]{0.45\textwidth}
			\includegraphics[width=\textwidth]{../figures/GDP_forecast.png}
			\caption{GDP: Out-of-sample forecast}
			\label{fig:gdp_out}
		\end{subfigure}
		
		% Trade Balance: In-sample and Forecast
		\begin{subfigure}[b]{0.45\textwidth}
			\includegraphics[width=\textwidth]{../figures/TradeBalance_fitted_vs_actual.png}
			\caption{Trade Balance: In-sample forecast}
			\label{fig:tb_in}
		\end{subfigure}
		\hfill
		\begin{subfigure}[b]{0.45\textwidth}
			\includegraphics[width=\textwidth]{../figures/TradeBalance_forecast.png}
			\caption{Trade Balance: Out-of-sample forecast}
			\label{fig:tb_out}
		\end{subfigure}
		
		% Exchange Rate: In-sample and Forecast
		\begin{subfigure}[b]{0.45\textwidth}
			\includegraphics[width=\textwidth]{../figures/ExchangeRate_fitted_vs_actual.png}
			\caption{Exchange Rate: In-sample forecast}
			\label{fig:er_in}
		\end{subfigure}
		\hfill
		\begin{subfigure}[b]{0.45\textwidth}
			\includegraphics[width=\textwidth]{../figures/ExchangeRate_forecast.png}
			\caption{Exchange Rate: Out-of-sample forecast}
			\label{fig:er_out}
		\end{subfigure}
		
		\begin{minipage}{\textwidth}
			\scriptsize
			\textit{Notes}: XX.
			
			\textit{Source}: Author's calculations using data from the UK Statistical Office.
		\end{minipage}
	\end{figure}
	
	Out-of-sample GDP forecasts (Panel b) suggest mean reversion to a stationary 
	level (drift = 1,832.99\textsuperscript{***}, SE = 232.75), with 95\% prediction 
	intervals widening to $\pm$6.7\% by 2025Q4. The Trade Balance model
	(ARIMA(0,1,4); Panels c--d) shows similar dynamics, passing residual 
	autocorrelation tests (Ljung-Box $p = 0.24$) but exhibiting volatility
	during Brexit negotiations (2019--2020). Exchange Rate forecasts (ARIMA(1,1,0); 
	Panels e--f) display narrower intervals ($\pm$2.1\% at $h=12$), consistent with 
	lower persistence ($\phi_1 = 0.259\textsuperscript{***}$). 
	
	\section{Multivariate Analysis}
	
	Note, a quick theoretical framework can be found in Appendix \ref{app}.
	
	\subsection{Optimal lag selection}
	
	We use the VARselect command in R to estimate the optimal lags. The results in 
	table \ref{fig:lagselection} present the optimal lags as suggested by four different tests. These
	tests aim to balance out the goodness-of-fit against model complexit with lower 
	values indicating better models. While AIC 
	(Akaike Information Criterion) and FPE (Final Prediction Error) impose a constant
	penalty per additional coefficient (and hence, lag) added to the model, they opt
	for more complex dynamics if that means reducing the residual variance. 
	By contrast, SC (Schwarz Information Criterion, also known as Bayesian Information Criterion) 
	and HQ (Hannan-Quinn) have penalties that increase with the number of lags used, 
	hence selecting only the most essential lags. 
	
	\begin{table}[ht]
		\centering
		\begin{threeparttable}
			\caption{\textsc{Optimal Lag Selection Based on Information Criteria}}\label{fig:lagselection}
			\begin{tabular}{lc}
				\\[-1.8ex]  \hline \hline  \\[-1.8ex] 
				{Criterion} & {Optimal Lag} \\
				\midrule
				Akaike Information Criterion (AIC) & 7 \\
				Hannan-Quinn Information Criterion (HQ)  & 1 \\
				Schwarz Criterion (SC) / Bayesian Information Criterion (BIC)  & 1 \\
				Final Prediction Error (FPE) & 7 \\
				\hline \hline  \\[-1.8ex] 
			\end{tabular}
		\end{threeparttable}
		\begin{minipage}{\textwidth}
			\footnotesize
			\textit{Notes:} Coefficient estimates (Coeff.) with standard errors (SE). 
			{--} = Component not included in model. Significance levels: $^{***}p<0.01$, $^{**}p<0.05$, $^{*}p<0.1$. 
			AR = Autoregressive term, MA = Moving Average term. GDP intercept in original units. 
			\textit{Source:} Author's calculations using UK Statistical Office data.
		\end{minipage}
	\end{table}
	This explains why AIC and FPE suggest
	7 lags while SC and HQ suggest only 1 lag. In the following, we go with AIC’s and 
	FPE’s results using 7 lags for our analysis. Our rationale is its widespread 
	use in the econometric modeling literature and its 
	robustness in capturing model fit while penalizing complexity less stringently 
	than SC or HQ. This makes AIC  particularly suitable for applications 
	where predictive accuracy is prioritized, as it allows for a slightly more 
	flexible model specification, which is often beneficial in a context 
	with potential structural nuances.
	
	\subsection{VAR Estimation}
	
	\paragraph*{Statistical model.}
	Our statistical model writes
	\begin{equation*}
		\bm{Y}_t = \bm{C} + \sum_{i=1}^7 \bm{\Phi}_i \bm{Y}_{t-i} + \bm{\varepsilon}_t
	\end{equation*}
	
	Where the first-differences endogenous variables write
	\begin{equation*}
		\bm{Y}_t = \begin{bmatrix} 
			\Delta y_t \\ 
			\Delta tb_t \\ 
			\Delta e_t 
		\end{bmatrix}, \quad
		\bm{Y}_{t-i} = \begin{bmatrix} 
			\Delta y_{t-i} \\ 
			\Delta tb_{t-i} \\ 
			\Delta e_{t-i} 
		\end{bmatrix}
	\end{equation*}
	
	The constant vector is 
	\begin{equation*}
		\bm{C} = \begin{bmatrix} 
			c_y \\ 
			c_{tb} \\ 
			c_e 
		\end{bmatrix}
	\end{equation*}
	
	We denote the lag coefficient matrices ($3 \times 3$ for each lag $i=1,...,7$):
	\begin{equation*}
		\bm{\Phi}_i = \begin{bmatrix}
			\phi^{yy}_i & \phi^{ytb}_i & \phi^{ye}_i \\
			\phi^{tby}_i & \phi^{tbtb}_i & \phi^{tbe}_i \\
			\phi^{ey}_i & \phi^{etb}_i & \phi^{ee}_i
		\end{bmatrix}
	\end{equation*}
	
	\begin{assumption}[Error Term Properties]\label{assump:errors}
		The innovation process satisfies:
		
		(i) Multivariate normality 
		\begin{equation*}
			\bm{\varepsilon}_t \overset{\text{i.i.d.}}{\sim} \mathcal{N}\left( \bm{0}, \bm{\Omega} \right)
		\end{equation*}
		
		(ii) The variance-covariance matrix is constant over time and given by:
		\begin{equation}\label{eq:omega}
			\bm{\Omega} = \begin{bmatrix}
				\sigma^2_y & \sigma_{y,tb} & \sigma_{y,e} \\
				\sigma_{tb,y} & \sigma^2_{tb} & \sigma_{tb,e} \\
				\sigma_{e,y} & \sigma_{e,tb} & \sigma^2_e
			\end{bmatrix}
		\end{equation}
		with $\mathbb{E}[\varepsilon_{j,t}\varepsilon_{k,s}] = 0$ for all $j \neq k$ or $t \neq s$ (orthogonal across equations and time).
	\end{assumption}
	
	\paragraph*{OLS estimation.} 
	
	
	The vector autoregression estimates reveal heterogeneous dynamics across the
	differenced variables. For first-differenced GDP (Column 1), significant 
	negative autocorrelation appears at Lag 1 ($-0.345^{***}$) and 
	Lag 5 ($-0.268^{**}$), while lagged trade balance shocks negatively
	affect GDP at Lag 2 ($-0.769^{***}$) and Lag 3 ($-0.695^{**}$). 
	The trade balance equation (Column 2) exhibits strong persistence with 
	significant negative own-lag effects (e.g., Lag 1: $-0.524^{***}$)
	and positive feedback from GDP at Lag 2 ($0.129^{**}$) and Lag 7 ($0.131^{**}$). 
	
	First-differenced exchange rates (Column 3) show limited explanatory power, 
	with only Lag 1 ($0.223^{**}$) achieving marginal significance. 
	The constant term is statistically significant only for GDP ($4,\!542.693^{***}$). 
	Model fit varies substantially across equations, with the trade balance 
	specification explaining 68.2\% of variance ($R^2 = 0.682$), compared to 
	15.0\% for exchange rates. Large standard errors for exchange rate coefficients 
	in GDP and trade balance equations (e.g., 43,296.580 for Lag 1) suggest
	limited precision in these estimates. All models use 104 quarterly observations.
	
	\begin{table}[!htbp] \centering 
		\caption{\textsc{Vector Autoregression OLS Model Estimates of GDP, Trade Balance, and Exchange Rate} }
		\label{tab:var_results} 
		\small 
		\resizebox{!}{0.2\textheight}{%
			\begin{tabular}{@{\extracolsep{5pt}}lccc} 
				\\[-1.8ex]\hline 
				\hline \\[-1.8ex] 
				& $\Delta$GDP & $\Delta$Trade Balance & $\Delta$Exchange Rate \\ 
				\\[-1.8ex] & (1) & (2) & (3)\\ 
				\hline \\[-1.8ex] 
				Lag 1 $\Delta$GDP & $-$0.345$^{***}$ (0.111) & $-$0.088$^{*}$ (0.046) & 0.000 (0.000) \\[1.2ex]  
				Lag 2 $\Delta$GDP & $-$0.234$^{*}$ (0.121) & 0.129$^{**}$ (0.050) & 0.000 (0.000) \\[1.2ex]  
				Lag 3 $\Delta$GDP & 0.064 (0.117) & $-$0.026 (0.049) & $-$0.000 (0.000) \\[1.2ex]  
				Lag 4 $\Delta$GDP & $-$0.158 (0.114) & $-$0.067 (0.048) & $-$0.000 (0.000) \\[1.2ex]  
				Lag 5 $\Delta$GDP & $-$0.268$^{**}$ (0.115) & 0.058 (0.048) & $-$0.000 (0.000) \\[1.2ex]  
				Lag 6 $\Delta$GDP & $-$0.163 (0.115) & $-$0.227$^{***}$ (0.048) & 0.000 (0.000) \\[1.2ex]  
				Lag 7 $\Delta$GDP & $-$0.190 (0.124) & 0.131$^{**}$ (0.052) & $-$0.000 (0.000) \\[1.2ex]  
				Lag 1 $\Delta$Trade Balance & $-$0.390 (0.248) & $-$0.524$^{***}$ (0.103) & $-$0.000 (0.000) \\[1.2ex]  
				Lag 2 $\Delta$Trade Balance & $-$0.769$^{***}$ (0.280) & $-$0.606$^{***}$ (0.117) & $-$0.000 (0.000) \\[1.2ex]  
				Lag 3 $\Delta$Trade Balance & $-$0.695$^{**}$ (0.337) & $-$0.640$^{***}$ (0.140) & 0.000 (0.000) \\[1.2ex]  
				Lag 4 $\Delta$Trade Balance & $-$0.914$^{**}$ (0.377) & $-$0.065 (0.157) & $-$0.000 (0.000) \\[1.2ex]  
				Lag 5 $\Delta$Trade Balance & $-$0.261 (0.328) & $-$0.101 (0.137) & $-$0.000 (0.000) \\[1.2ex]  
				Lag 6 $\Delta$Trade Balance & $-$0.169 (0.293) & 0.060 (0.122) & $-$0.000 (0.000) \\[1.2ex]  
				Lag 7 $\Delta$Trade Balance & $-$0.059 (0.229) & 0.295$^{***}$ (0.096) & $-$0.000 (0.000) \\[1.2ex]  
				Lag 1 $\Delta$Exchange Rate & 17,108.830 (43,296.580) & $-$7,168.638 (18,050.790) & 0.223$^{**}$ (0.110) \\[1.2ex]  
				Lag 2 $\Delta$Exchange Rate & $-$49,988.790 (44,212.210) & $-$4,433.856 (18,432.530) & 0.003 (0.112) \\[1.2ex]  
				Lag 3 $\Delta$Exchange Rate & 82,708.040$^{*}$ (43,869.740) & $-$6,702.783 (18,289.750) & 0.165 (0.111) \\[1.2ex]  
				Lag 4 $\Delta$Exchange Rate & 2,302.463 (44,349.430) & $-$9,484.912 (18,489.740) & $-$0.075 (0.112) \\[1.2ex]  
				Lag 5 $\Delta$Exchange Rate & $-$7,247.033 (44,087.360) & $-$15,840.930 (18,380.480) & $-$0.150 (0.112) \\[1.2ex]  
				Lag 6 $\Delta$Exchange Rate & $-$14,410.070 (44,051.490) & $-$91.898 (18,365.530) & $-$0.014 (0.112) \\[1.2ex]  
				Lag 7 $\Delta$Exchange Rate & 22,356.620 (42,538.090) & $-$588.238 (17,734.570) & 0.091 (0.108) \\[1.2ex]  
				Constant & 4,542.693$^{***}$ (1,716.036) & $-$380.569 (715.433) & $-$0.001 (0.004) \\[1.4ex]  
				Observations & 104 & 104 & 104 \\ 
				R$^{2}$ & 0.314 & 0.682 & 0.150 \\ 
				Adjusted R$^{2}$ & 0.138 & 0.601 & $-$0.067 \\ 
				\hline \\[-1.8ex] 
			\end{tabular} 
		}
		\begin{minipage}{\textwidth}
			\footnotesize
			\textit{Notes:} Significance levels: $^{***}p<0.01$, $^{**}p<0.05$, $^{*}p<0.1$. Standard errors in parenthesis.
			
			\textit{Source:} Author's calculations using UK Statistical Office data.
		\end{minipage}
	\end{table} 
	
	
	
	
	\subsection{Residual testing}
	
	The residual diagnostics in Table \ref{tab:diagnostics} indicate mixed 
	properties of the VAR model. While the Portmanteau test for serial 
	correlation rejects the null hypothesis of no autocorrelation at the 1\% 
	significance level ($p = 0.009$), we fail to reject the absence of ARCH 
	effects ($p = 0.440$), suggesting no persistent conditional heteroskedasticity.
	The Jarque-Bera test overwhelmingly rejects normality ($p < 0.001$), implying 
	residuals exhibit substantial non-Gaussian features such as heavy tails or skewness. 
	While non-normality invalidates standard $t$-tests in small samples, the central 
	limit theorem provides asymptotic justification for inference in large samples
	like ours ($T = 104$). Robust standard errors or bootstrap methods remain advisable
	for hypothesis testing
	
	\begin{table}[!htbp]
		\centering
		\caption{\textsc{Residual Diagnostic Tests}}
		\label{tab:diagnostics}
		\small
		\begin{tabular}{@{}lSSS@{}}
			\\  \hline \hline \\
			\textbf{Test} & \textbf{Statistic} & \textbf{df} & \textbf{$p$-value} \\ 
			\midrule
			Serial Correlation (Portmanteau) & 70.64 & 45 & 0.009 \\
			ARCH Effects & 290.95 & 288 & 0.440 \\
			Normality (Jarque-Bera) & 7261.26 & 6 & <0.001 \\
			\hline \hline  \\
		\end{tabular}
	\end{table}
	
	\begin{assumption}[Refinement on the Error Term Structure]\label{assump:errors}
		The residual vector $\bm{\varepsilon}_t$ satisfies: 
		
		(i) Mean independence: $\mathbb{E}[\bm{\varepsilon}_t] = 
		\bm{0}$ and $\mathbb{E}[\bm{\varepsilon}_t|\mathcal{F}_{t-1}] = \bm{0}$ ( where 
		$\mathcal{F}_{t-1}$ is the information set containing all variables $\{\bm{Y}_{t-1}, \bm{Y}_{t-2}, \dots\}$)
		
		(ii) Homoskedasticity: $\text{Var}(\bm{\varepsilon}_t) = \bm{\Omega}$ where 
		$\bm{\Omega} = \begin{bmatrix}
			\sigma^2_y & \sigma_{y,tb} & \sigma_{y,e} \\
			\sigma_{tb,y} & \sigma^2_{tb} & \sigma_{tb,e} \\
			\sigma_{e,y} & \sigma_{e,tb} & \sigma^2_e
		\end{bmatrix}$ is constant 
		
		(iii) Weak exogeneity: $\varepsilon_{j,t} \perp \varepsilon_{k,s}$ for $j \neq k$ or $t \neq s$ \\
		Residual diagnostics (Table~\ref{tab:diagnostics}) reveal non-normal innovations but no ARCH effects.
		Asymptotic normality of estimators holds for $T=104$ under the Lindeberg-Feller version of the Central Limit Theorem.
	\end{assumption}
	
	
	\subsection{VAR forecasts}
	
	Our out-of-sample forecasts for $\Delta$ GDP remain tightly centred on zero—consistent 
	with the stationary behavior we established in our VAR—while the fan chart’s 
	gradually widening bands reflect the growing uncertainty as the forecast 
	horizon extends. Importantly, even after the dramatic COVID-19 shock in early 
	2020, the model shows a rapid return of quarterly GDP growth to its long-run 
	average of essentially zero, underscoring both the economy’s resilience and 
	the appropriateness of our stationary specification.
	
	Our out-of-sample forecasts for the change in trade balance exhibit a stable trajectory 
	centered near zero, consistent with the mean-reverting dynamics implied by our 
	stationary VAR specification. While the confidence bands widen modestly
	over the 8-quarter horizon—reflecting inherent uncertainty in trade flow
	dynamics—the forecast intervals remain contained, suggesting limited long-term 
	deviation from equilibrium. Notably, despite the unprecedented trade volatility 
	during the COVID-19 pandemic, the model projects a rapid reversion to pre-shock 
	trends, underscoring the self-correcting mechanisms embedded in trade balance 
	adjustments under stationary conditions.
	
	\begin{figure}[!htbp]
		\centering
		\caption{\textsc{VAR Forecasts Performance}}
		\label{fig:var_forecasts}
		
		\begin{subfigure}[b]{0.32\textwidth}
			\centering
			\includegraphics[width=\textwidth]{../figures/VAR_forecast_GDP.png}
			\caption{$\Delta$ GDP}
			\label{fig:forecast_gdp}
		\end{subfigure}
		\\
		\begin{subfigure}[b]{0.32\textwidth}
			\centering
			\includegraphics[width=\textwidth]{../figures/VAR_forecast_TradeBalance.png}
			\caption{$\Delta$ Trade Balance}
			\label{fig:forecast_tb}
		\end{subfigure}
		\begin{subfigure}[b]{0.32\textwidth}
			\centering
			\includegraphics[width=\textwidth]{../figures/VAR_forecast_ExchangeRate.png}
			\caption{$\Delta$ Exchange Rate}
			\label{fig:forecast_fx}
		\end{subfigure}
		
		\begin{minipage}{\textwidth}
			\footnotesize
			\textit{Notes:} Fan charts show 8-period ahead VAR forecasts with 95\% confidence bands. 
			Shaded regions represent forecast uncertainty. 
			
			\textit{Sources:} UK Statistical Office.
		\end{minipage}
	\end{figure}
	
	The exchange rate forecasts display marginally wider confidence bands compared 
	to GDP and trade balance, reflecting the well-documented volatility of financial 
	variables. Nevertheless, the central forecast path gravitates toward 
	zero—aligning with purchasing power parity fundamentals and the stationarity of 
	our VAR system. Even after accounting for the sharp exchange rate fluctuations 
	observed during recent crises (e.g., post-COVID dollar shortages), the model 
	anticipates a gradual return to equilibrium, highlighting the stabilizing role 
	of monetary policy and international arbitrage in anchoring expectations over extended horizons.
	
	\subsection{Cholesky decomposition}
	
	The Cholesky decomposition factorizes the residual covariance matrix of our 
	VAR into a lower‐triangular matrix, which yields a set of orthogonal
	(i.e. uncorrelated) structural shocks. By imposing a recursive identification
	scheme—where the first variable is assumed to be contemporaneously exogenous, 
	the second may respond immediately to the first, the third to the first two,
	and so on—we obtain a uniquely defined causal ordering. This ordering lets us 
	interpret each impulse‐response function as the dynamic effect of a one‐unit 
	shock in one variable on all others in the system. In our analysis, we apply 
	the Cholesky decomposition to the VAR residuals and adopt the 
	ordering: $\Delta$GDP \textrightarrow $\Delta$Exchange Rate \textrightarrow $\Delta$Trade Balance, 
	so that shocks to  $\Delta$GDP are treated as exogenous and shocks to $\Delta$Trade Balance 
	may reflect contemporaneous feedback from both $\Delta$GDP and $\Delta$Exchange Rate.
	
	\subsection{Impulse response functions}
	In the quarterly VAR we treat the \emph{exchange rate} as the fastest‐moving
	variable, priced continuously in financial markets; \emph{GDP} as slower,
	because it aggregates many real‐economy decisions and is only observed with a
	quarterly lag; and the \emph{trade balance} as the slowest, constrained by
	shipping lags and fixed-term contracts.  Hence our preferred Cholesky ordering
	is $\Delta E \rightarrow \Delta Y \rightarrow \Delta TB$.
	
	The resulting impulse-response functions (Fig. 4) show:
	
	\begin{enumerate}
		\item A one-quarter exchange-rate depreciation produces an immediate
		trade-balance deficit (J-curve) and a modest, lagged GDP fall as dearer
		imports squeeze real income.
		\item Trade-balance shocks hardly move GDP, suggesting that capital-account
		inflows cushion real activity.
		\item GDP innovations elicit only muted exchange-rate reactions, consistent
		with Mundell–Fleming under monetary autonomy.
	\end{enumerate}
	
	All responses fade within eight quarters and lie inside 95 \% confidence
	bands, confirming stationarity and the plausibility of the financial-markets-first
	identification strategy.
	
	
	
	\begin{figure}[!htbp]
		\centering
		\caption{\textsc{Impulse Response Functions}}
		\label{fig:irf1}
		
		\begin{subfigure}[b]{0.32\textwidth}
			\centering
			\includegraphics[width=\textwidth]{../figures/IRF_plots/IRF_to_First.differenced.GDP.png}
			\caption{Response to $\Delta$ GDP shock}
			\label{fig:irf_gdp}
		\end{subfigure}
		\\
		\begin{subfigure}[b]{0.32\textwidth}
			\centering
			\includegraphics[width=\textwidth]{../figures/IRF_plots/IRF_to_First.differenced.exchange.rate.png}
			\caption{Response to $\Delta$ Trade Balance shock}
			\label{fig:irf_tb}
		\end{subfigure}
		\begin{subfigure}[b]{0.32\textwidth}
			\centering
			\includegraphics[width=\textwidth]{../figures/IRF_plots/IRF_to_First.differenced.balance.of.payments.png}
			\caption{Response $\Delta$ Exchange Rate shock}
			\label{fig:irf_fx}
		\end{subfigure}
		
		\begin{minipage}{\textwidth}
			\footnotesize
			\textit{Notes:} Confidence bands: 95\% intervals computed via Monte Carlo simulations (1,000 repetitions, residual-based bootstrapping).
			Identification: Orthogonalized shocks using Cholesky decomposition with variable ordering: 
			[Exchange Rate $\rightarrow$ GDP $\rightarrow$ Trade Balance ]. Responses shown over 8-quarter horizon, 
			consistent with VAR forecast period.
			Zero line (horizontal axis) represents long-run equilibrium.
			
			
			\textit{Sources:} UK Statistical Office.
		\end{minipage}
	\end{figure}
	
	Over an 8-quarter horizon, these IRFs reveal how shocks propagate
	through the economy: for instance, a positive innovation to GDP induces a 
	short-lived boost to trade balance, while exchange rate responses exhibit 
	persistent oscillations consistent with overshooting dynamics. The gradual 
	convergence of all variables toward zero underscores the stationarity of our 
	VAR specification, as shocks dissipate without permanent effects.
	
	\subsection{Robustness analysis: modify the ordering of the variables}
	
	Figure \ref{fig:irf2} presents the impulse response functions (IRFs) derived from our VAR model, 
	under an alternative Cholesky ordering
	illustrating the dynamic responses of the system’s variables—first-differenced GDP, 
	trade balance, and exchange rate—to orthogonalized structural shocks. The 
	shaded regions represent 95\% confidence intervals generated via Monte Carlo 
	simulations, quantifying the statistical uncertainty around the median response
	trajectories. 
	
	\begin{figure}[!htbp]
		\centering
		\caption{\textsc{Impulse Response Functions}}
		\label{fig:irf2}
		
		\begin{subfigure}[b]{0.32\textwidth}
			\centering
			\includegraphics[width=\textwidth]{../figures/IRF_plots2/IRF_to_First.differenced.GDP.png}
			\caption{Response to $\Delta$ GDP shock}
			\label{fig:irf_gdp2}
		\end{subfigure}
		\\
		\begin{subfigure}[b]{0.32\textwidth}
			\centering
			\includegraphics[width=\textwidth]{../figures/IRF_plots2/IRF_to_First.differenced.exchange.rate.png}
			\caption{Response to $\Delta$ Trade Balance shock}
			\label{fig:irf_tb2}
		\end{subfigure}
		\begin{subfigure}[b]{0.32\textwidth}
			\centering
			\includegraphics[width=\textwidth]{../figures/IRF_plots2/IRF_to_First.differenced.balance.of.payments.png}
			\caption{Response $\Delta$ Exchange Rate shock}
			\label{fig:irf_fx2}
		\end{subfigure}
		
		\begin{minipage}{\textwidth}
			\footnotesize
			\textit{Notes:} Confidence bands: 95\% intervals computed via Monte Carlo simulations (1,000 repetitions, residual-based bootstrapping).
			Identification: Orthogonalized shocks using Cholesky decomposition with variable ordering: 
			[$\Delta$\textit{GDP} $\rightarrow$ $\Delta$\textit{Balance of Payments} $\rightarrow$ $\Delta$\textit{Exchange Rate}].
			Responses shown over 8-quarter horizon, 
			consistent with VAR forecast period.
			Zero line (horizontal axis) represents long-run equilibrium.
			
			
			\textit{Sources:} UK Statistical Office.
		\end{minipage}
	\end{figure}
	
	%\begin{figure}
	
	%{\centering \includegraphics[width=0.8\linewidth]{../figures/IRF_plots2/IRF_to_First.differenced.GDP} 
		
		%}
	
	%\caption{GDP - Shock reactions}\label{fig:unnamed-chunk-27}
	%\end{figure}
	
	%The impulse response functions for Trade Balance are depicted as:
	
	%\begin{figure}
	
	%{\centering \includegraphics[width=0.8\linewidth]{../figures/IRF_plots2/IRF_to_First.differenced.balance.of.payments} 
		
		%}
	
	%\caption{Trade Balance - Shock reactions}\label{fig:unnamed-chunk-28}
	%\end{figure}
	
	%The impulse response functions for Exchange Rate are depicted as:
	
	%\begin{figure}
	
	%{\centering \includegraphics[width=0.8\linewidth]{../figures/IRF_plots2/IRF_to_First.differenced.exchange.rate} 
		
		%}
	
	%\caption{Exchange Rate - Shock reactions}\label{fig:unnamed-chunk-29}
	%\end{figure}
	
	\section{Concluding Remarks}
	This study analyzed the UK’s GDP, trade balance, and exchange rate dynamics using univariate and multivariate econometric frameworks. 
	First-differencing addressed non-stationarity, enabling robust modeling. ARIMA results showed distinct dynamics:
	GDP combined trend persistence with shock absorption, trade balance exhibited error correction, and exchange rates had moderate momentum.
	VAR analysis underscored interconnectedness: exchange rate shocks triggered
	J-curve trade effects, while GDP shocks had delayed spillovers. 
	The UK’s resilience post-COVID and Brexit highlighted inherent stabilization.
	
	An implication for policymakers would be to prioritize lagged effects of exchange rates and GDP shocks during crises.
	
	In terms of limitations, our work exhibits sensitivity to VAR ordering and potential overfitting. 
	To extend the analysis, we could have integrated structural VARs with long-run restrictions, or machine 
	learning methods to improve accuracy.
	
	\newpage
	
	\appendix
	
	\section{Multivariate Analysis - theoretical framework}
	\label{app}
	
	
	
	
	This section develops a theoretical framework to analyze the dynamic interactions 
	among GDP, exchange rates, and trade balance in the UK from 1955 to 2024, incorporating 
	the COVID-19 pandemic (2020--2021) and Brexit (2016--2024) as exogenous shocks. 
	The model synthesizes New Keynesian open-economy dynamics %\citep{gali2005}
	, financial frictions
	%\citep{cespedes2004}
	, Mundell-Fleming trilemma constraints %\citep{mundell1963,fleming1962}
	, 
	and enhanced exchange rate pass-through %\citep{obstfeld1995}
	. It provides a foundation for 
	Vector Autoregression (VAR) analysis by proposing a Cholesky ordering that reflects theoretical causality.
	Table \ref{tab:notations} summarizes the model’s variables and assumptions.
	
	\begin{table}[ht]
		\resizebox{!}{0.1\textheight}{%
			\centering
			\caption{\textsc{Notations and Assumptions} }
			\label{tab:notations}
			\begin{tabular}{p{0.45\textwidth} p{0.45\textwidth}}
				\toprule
				\textbf{Notations} & \textbf{Assumptions} \\
				\midrule
				Output gap (\( y_t \)): Log GDP deviations. & Small open economy with imperfect capital mobility. \\[1.5ex] 
				GBP Nominal exchange rate (\( E_t \)): & Calvo-style price stickiness %\citep{gali2005}. 
				\\
				Trade balance (\( TB_t \)) & Fixed (pre-1992) or floating (post-1992) exchange rate regimes. \\[1.5ex] 
				Inflation (\( \pi_t \)): Domestic CPI inflation & Open capital account with risk premiums %\citep{engel2016}.
				\\
				Interest rate (\( r_t \)): Bank of England policy rate & Exogenous shocks from COVID-19 and Brexit %\citep{ons2023,obr2025}.
				\\ 
				Fiscal policy (\( g_t \)): Government spending & \\
				COVID-19 shock (\( \xi_t^{COV} \)): Temporary AR(1) shock, \( \xi_t^{COV} = \rho^{COV} \xi_{t-1}^{COV} + \epsilon_t^{COV} \), \( \rho^{COV} < 1 \). & \\
				Brexit shock (\( \tau_t^{BRX} \)): Persistent AR(1) shock, \( \tau_t^{BRX} = \rho^{BRX} \tau_{t-1}^{BRX} + \epsilon_t^{BRX} \), \( \rho^{BRX} \approx 1 \). & \\
				\bottomrule
			\end{tabular}
		}
	\end{table}
	
	The model comprises six core equations, derived below with explicit integration of COVID-19 and Brexit shocks.
	
	\paragraph*{New-Keynesian IS curve} \\
	
	Log-linearizing the household Euler equation links output to its expected future value and the gap between the real policy rate and the natural rate. Decomposing aggregate demand by steady-state expenditure shares, investment is allowed to fall with the interest rate and Brexit uncertainty. Adding open-economy terms introduces the expected change in the real exchange rate and foreign output, while fiscal spending enters with a (potentially larger) multiplier under fixed exchange rates. Two additive shocks capture demand losses from COVID-19 and Brexit, yielding
	
	\begin{equation}
		y_t = E_t y_{t+1}
		- \frac{1}{\sigma}\bigl(r_t - E_t\pi_{t+1} - r_t^{n}\bigr)
		+ \alpha\bigl(E_t q_{t+1} - q_t\bigr)
		+ \eta,y_t^{*}
		+ \mu,g_t
		- \delta^{\mathrm{COV}}\xi_t^{\mathrm{COV}}
		- \delta^{\mathrm{BRX}}\tau_t^{\mathrm{BRX}}.
	\end{equation}
	
	\paragraph*{New-Keynesian Phillips Curve}
	Calvo pricing implies that only a share \(1-\theta\) of firms can reset prices each period.
	\footnote{The remaining share \(\theta\) keep their prices fixed, generating nominal rigidity.} \\
	Log-linearising the optimal pricing condition and the aggregate price index yields a forward-looking Phillips curve. \\
	Marginal cost depends on the output gap and the real exchange rate, while open-economy pass-through adds the change in the nominal exchange rate. \\
	Exogenous terms capture COVID-19 supply bottlenecks and post-Brexit import-cost pressures. \\
	The resulting log-linear Phillips curve is
	\begin{equation}
		\pi_t
		= \beta\,E_t\pi_{t+1}
		+ \kappa\,y_t
		+ \lambda\bigl(q_t - q_{t-1}\bigr)
		+ \nu\,\Delta E_t
		+ \gamma^{\mathrm{COV}}\xi_t^{\mathrm{COV}}
		+ \gamma^{\mathrm{BRX}}\tau_t^{\mathrm{BRX}},
		\label{eq:phillips_condensed}
	\end{equation}
	where \(\kappa=(1-\theta)(1-\beta\theta)/\theta\).
	The coefficients \(\lambda\) and \(\nu\) quantify real-exchange-rate and pass-through effects, while \(\gamma^{\mathrm{COV}}\) and \(\gamma^{\mathrm{BRX}}\) capture the inflationary impact of pandemic-related supply constraints and Brexit-induced trade frictions.
	
	
	\paragraph*{Trade‐Balance Equation}%
	Exports and imports depend on the real exchange rate and foreign–domestic demand: log‐linearising \(X_t=(Q_t)^{-\eta_x}Y_t^{*}\) and \(M_t=(Q_t)^{\eta_m}Y_t\) gives \(tb_t=\eta_x q_t+\gamma_y y_t^{*}-\gamma_m y_t\). \\
	Allowing for a J-curve lag and exchange-rate pass-through, and adding COVID-19 trade disruptions and Brexit-related EU trade losses, yields
	\begin{equation}
		tb_t
		= \gamma_x\bigl(q_t-\phi q_{t-1}\bigr)
		+ \gamma_y\,y_t^{*}
		- \gamma_m\,y_t
		+ \chi\,\Delta E_t
		- \psi^{\mathrm{COV}}\xi_t^{\mathrm{COV}}
		- \psi^{\mathrm{BRX}}\tau_t^{\mathrm{BRX}},
		\label{eq:tb_condensed}
	\end{equation}
	where \(\gamma_x=\eta_x\), \(\phi\in(0,1)\) captures the J-curve, \(\chi\) measures pass-through, and the \(\psi\)-coefficients quantify pandemic and Brexit shocks to net exports.
	
	\paragraph{Exchange-Rate Dynamics}
	Under a fixed‐exchange‐rate regime (e.g.\ pre-1992 ERM membership), the nominal rate is constant at its peg \(\bar{E}\).%
	For floating regimes, uncovered-interest parity—augmented with a risk premium \(\rho_t\) and short-run deviations \(\psi_t\)—links the spot rate to expected future depreciation:
	\[
	E_t\,=\,
	E_t E_{t+1}\,\frac{1+r_t}{1+r_t^{*}+\rho_t}\;+\;\psi_t.
	\]
	Combining both cases, the exchange-rate rule is
	\begin{equation}
		E_t \;=\;
		\begin{cases}
			\bar{E}, &
			\text{(fixed regime)}\\[4pt]
			E_t E_{t+1}\,\dfrac{1+r_t}{1+r_t^{*}+\rho_t}\;+\;\psi_t, &
			\text{(floating regime).}
		\end{cases}
		\label{eq:uip_condensed}
	\end{equation}
	Here \(r_t\) and \(r_t^{*}\) denote domestic and foreign policy rates, while \(\rho_t\) captures the risk premium consistent with the exchange-rate trilemma.
	
	\paragraph*{Balance Sheet Effects}
	
	Investment faces financial frictions.
	
	Investment
	
	\begin{equation*}
		I_t = I(r_t, \theta_t)
	\end{equation*}
	
	Financial conditions
	
	\begin{equation*}
		\theta_t = \theta_0 - \zeta (E_t D_t^* + \tau_t^{BRX})
	\end{equation*}
	
	\begin{equation}
		I_t = I(r_t, \theta_t), \quad \theta_t = \theta_0 - \zeta (E_t D_t^* + \tau_t^{BRX})
		\label{eq:bs}
	\end{equation}
	
	\paragraph*{Monetary Policy}
	
	The central bank sets the interest rate. Fixed regimes
	
	\begin{equation*}
		r_t = r_t^* + \rho_t + \epsilon_t
	\end{equation*}
	
	Floating regimes (Taylor rule)
	
	\begin{equation*}
		r_t = r_t^n + \phi_\pi \pi_t + \phi_y y_t + \epsilon_t
	\end{equation*}
	
	Therefore 
	
	\begin{equation}
		r_t =
		\begin{cases} 
			r_t^* + \rho_t + \epsilon_t & \text{if fixed regime} \\
			r_t^n + \phi_\pi \pi_t + \phi_y y_t + \epsilon_t & \text{if floating regime}
		\end{cases}
		\label{eq:taylor}
	\end{equation}
	
	\paragraph*{Implications for our analysis.} 
	The framework suggests the VAR ordering: 
	\( \xi_t^{COV} \rightarrow \tau_t^{BRX} \rightarrow E_t \rightarrow \pi_t \rightarrow y_t \rightarrow TB_t \rightarrow r_t \), reflecting:
	an exogenous, temporary COVID-19 shock %\citep{ons2023}.
	, A persistent and structural Brexit shock %\citep{obr2025}.
	, fixed exchange rate %\citep{dornbusch1976,mundell1963}.
	, inflation transmits shocks,% \citep{obstfeld1995}.
	output is affected by shocks,% \citep{fleming1962}.
	trade balance is endogenous,% \citep{backus1994}.
	and interest rate is constrained or reactive.% \citep{taylor1993}.
	
	
	
\end{document}
