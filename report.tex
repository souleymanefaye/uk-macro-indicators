% Options for packages loaded elsewhere
\PassOptionsToPackage{unicode}{hyperref}
\PassOptionsToPackage{hyphens}{url}
%
\documentclass[
]{article}
\usepackage{amsmath,amssymb}
\usepackage{iftex}
\ifPDFTeX
  \usepackage[T1]{fontenc}
  \usepackage[utf8]{inputenc}
  \usepackage{textcomp} % provide euro and other symbols
\else % if luatex or xetex
  \usepackage{unicode-math} % this also loads fontspec
  \defaultfontfeatures{Scale=MatchLowercase}
  \defaultfontfeatures[\rmfamily]{Ligatures=TeX,Scale=1}
\fi
\usepackage{lmodern}
\ifPDFTeX\else
  % xetex/luatex font selection
\fi
% Use upquote if available, for straight quotes in verbatim environments
\IfFileExists{upquote.sty}{\usepackage{upquote}}{}
\IfFileExists{microtype.sty}{% use microtype if available
  \usepackage[]{microtype}
  \UseMicrotypeSet[protrusion]{basicmath} % disable protrusion for tt fonts
}{}
\makeatletter
\@ifundefined{KOMAClassName}{% if non-KOMA class
  \IfFileExists{parskip.sty}{%
    \usepackage{parskip}
  }{% else
    \setlength{\parindent}{0pt}
    \setlength{\parskip}{6pt plus 2pt minus 1pt}}
}{% if KOMA class
  \KOMAoptions{parskip=half}}
\makeatother
\usepackage{xcolor}
\usepackage[margin=1in]{geometry}
\usepackage{graphicx}
\makeatletter
\def\maxwidth{\ifdim\Gin@nat@width>\linewidth\linewidth\else\Gin@nat@width\fi}
\def\maxheight{\ifdim\Gin@nat@height>\textheight\textheight\else\Gin@nat@height\fi}
\makeatother
% Scale images if necessary, so that they will not overflow the page
% margins by default, and it is still possible to overwrite the defaults
% using explicit options in \includegraphics[width, height, ...]{}
\setkeys{Gin}{width=\maxwidth,height=\maxheight,keepaspectratio}
% Set default figure placement to htbp
\makeatletter
\def\fps@figure{htbp}
\makeatother
\setlength{\emergencystretch}{3em} % prevent overfull lines
\providecommand{\tightlist}{%
  \setlength{\itemsep}{0pt}\setlength{\parskip}{0pt}}
\setcounter{secnumdepth}{5}
\usepackage{caption}
\usepackage{multirow}
\usepackage{float}
\restylefloat{table}
\let\oldtable\table
\let\endoldtable\endtable
\renewenvironment{table}[1][H]{\oldtable[H]}{\endoldtable}
\usepackage{fancyhdr}
\pagestyle{fancy}
\fancyhf{}
\fancyhead[L]{\textbf{2025-05-23}}
\fancyhead[C]{\textbf{Econometrics Problem Set 2}}
\fancyhead[R]{\textbf{Gereon Staratschek}}
\fancyfoot[C]{\thepage}
\usepackage{booktabs}
\usepackage{longtable}
\usepackage{array}
\usepackage{multirow}
\usepackage{wrapfig}
\usepackage{float}
\usepackage{colortbl}
\usepackage{pdflscape}
\usepackage{tabu}
\usepackage{threeparttable}
\usepackage{threeparttablex}
\usepackage[normalem]{ulem}
\usepackage{makecell}
\usepackage{xcolor}
\ifLuaTeX
  \usepackage{selnolig}  % disable illegal ligatures
\fi
\usepackage{bookmark}
\IfFileExists{xurl.sty}{\usepackage{xurl}}{} % add URL line breaks if available
\urlstyle{same}
\hypersetup{
  pdftitle={Macroeconometrics -- UK variables report},
  pdfauthor={Souleymane Faye \& Gereon Staratschek},
  hidelinks,
  pdfcreator={LaTeX via pandoc}}

\title{Macroeconometrics -- UK variables report}
\author{Souleymane Faye \& Gereon Staratschek}
\date{2025-05-23}

\begin{document}
\maketitle

\section*{Preliminaries}

In this project, we aim to analyze the development of the Gross Domestic
Product (GDP), exchange rates, and trade balance of the United Kingdom
(UK). We collect our data from the OECD open data portal. We use data on
a quarterly basis, starting latest in 1997. This period covers important
financial events such as the Global Financial Crisis (GFC) in 2007, the
Brexit referendum in 2016 and UK's final EU leave in 2020 as well as the
COVID-19 pandemic from 2020-2022.

\section*{Exercise 1}

\subsection*{Part 1: Analyzing the time series in levels}

Looking at the plain time series data and analyzing the plots, we get
the following results:

\begin{figure}

{\centering \includegraphics[width=0.8\linewidth]{../figures/uk_levels_base_plot} 

}

\caption{Residual diagnostics from ARIMA(1,0,4).}\label{fig:resid-plot}
\end{figure}

\textbf{GDP}: Data on UK's GDP is available on a quarterly basis
starting in 1955. GDP exhibits a clear, upwards trend with only a few
shocks (such as the 2020 COVID-19 oandemic or the 2007 financial crisis)
interrupting the general trend. Hence, the GDP of the UK is clearly not
stationary. However, the trend seems to be linear, suggesting that the
first-differences time series of the GDP might be stationary.
\textbackslash{}

\textbf{Trade Balance}: As GDP, data on the trade balance is available
on a quarterly basis starting in 1955. We see that the trade balance
fluctuates around 0 until the 1990s where a negative trend seems to set
in continueing until the 2010s, when trade balance starts fluctuating
around a low, negative value. However, since the observed spikes are
getting much bigger over time, we also see an increase in fluctuation
around the respective stationary mean. Hence, the data seems to be
stationary in the beginning and in the end with a negative trend being
observed between the 1990s and the 2010s. \textbackslash{}

\textbf{Exchange Rate}: Data on the exchange rate is available since
1997 on a quarterly basis. It exhibits stationarity between 1997 and
2007, as well as from 2007 onwards. In 2007, a shock seems to have
shifted the mean of the stationary process downwards.

\subsection*{Part 2: Conducting unit roots and stationarity tests}

\end{document}
